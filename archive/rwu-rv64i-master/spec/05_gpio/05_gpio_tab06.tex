
Table \ref{tab:asGpioPer06} shows the input and outputs of the GPIO Top.
\begin{table}[H]
\caption{GPIO Top}
\label{tab:asGpioPer06}
\centering
\begin{tabularx}{\textwidth}{|l |l |l |X|}
  \hline
  Pin & Direction & Width & Explanation \\
  \hline
  \hline
  clk\_i & in & 1 & System clock \\
  \hline
  rst\_i & in & 1 & System reset. Active high. \\
  \hline
  wbdAddr\_i(3:0) & in & 4 &  From WB-Bus. Peripherals addresses. \\
  \hline
  wbdDat\_i(63:0) & in & 64 &  From WB-Bus. Peripherals data input. \\
  \hline
  wbdDat\_o(63:0) & out & 64 &  To WB-Bus. Peripherals data output. \\
  \hline
  wbdWe\_i & in & 1 &  From WB-Bus. Peripherals write enable. \\
  \hline
  wbdSel\_i(7:0) & in & 8 &  From WB-Bus. Peripherals byte select. \\
  \hline
  wbdStb\_i & in & 1 &  From WB-Bus. Valid bus cycle.  \\
  \hline
  wbdAck\_o & out & 1 &  To WB-Bus. Error free bus transaction.  \\
  \hline
  wbdCyc\_i & in & 1 &  From WB-Bus. High for complete bus cycle.  \\
  \hline
  gpio\_o(7:0) & out & 8 &  To chip pins. GPIO outputs for one block. \\
  \hline
  gpioAdr\_(3:0) & out & 4 & To chip pins. Address for external GPIO devices. \\
  \hline
  cs\_o & out & 1 &  To chip pins. Chip select for outside GPIO blocks.   \\
  \hline
\end{tabularx}
\end{table}

\chapter{Introduction}
\label{ch:introduction}

\section{Motivation}
The increasing adoption of open-source Instruction Set Architectures (ISAs), such as RISC-V, has significantly transformed the landscape of embedded system and processor development.  
Unlike proprietary ISAs, RISC-V provides a modular and license-free foundation that encourages academic and industrial innovation in processor design, compiler construction, and system-level integration~\cite{riscv-spec}.  
As a result, universities and research institutes can design, implement, and study customized processors while maintaining compatibility with standardized software tools.

The RWU-RV64I processor, developed at the Hochschule Ravensburg-Weingarten, is a 64-bit implementation of the base RISC-V integer instruction set (RV64I).  
It serves as an educational and research platform for studying microarchitecture design, hardware–software co-design, and compiler interaction.  
However, the processor’s practical usability depends strongly on the availability of a reliable software toolchain.  
Without a well-integrated compiler, assembler, linker, and debugging environment, processor programming is limited to low-level assembly testing, restricting scalability and educational use.

Developing a complete and robust toolchain for the RWU-RV64I is therefore essential to enable C and C++ application development, facilitate efficient testing, and improve productivity in hardware verification and embedded systems research.  
The toolchain forms the central interface between software developers and hardware designers, bridging the gap between algorithmic implementation and instruction-level execution on the target core.  
Moreover, a standardized build process and integrated development environment allow future students and researchers to rapidly develop, compile, and test programs on the RWU-RV64I, fostering a sustainable development ecosystem within the university.

\section{Problem Statement and Scope}

The existing RWU-RV64I environment provides only a minimal \texttt{Makefile}-based setup with limited support for assembly-level test programs.  
While this setup suffices for verifying basic instruction functionality, it lacks the capabilities required for structured C program development and systematic evaluation of compiler optimizations.  
Furthermore, the absence of integration with modern development environments significantly restricts the usability, scalability, and maintainability of the system.

This thesis therefore focuses on designing and implementing a complete software toolchain for the RWU-RV64I processor that enables seamless compilation, linking, and execution of C programs on the target platform.  
Specifically, the work covers the following aspects:
\begin{itemize}
  \item Configuration and integration of the compiler, assembler, and linker components from the GNU toolchain (GCC and Binutils) for the RV64I architecture~\cite{gcc}.
  \item Development of a customized linker script and startup code adapted to the RWU-RV64I memory architecture and hardware configuration.
  \item Integration of the build environment into Eclipse to provide a user-friendly graphical interface for software development.
  \item Deployment and verification of the compiled binaries on the Zybo Zynq-7000 FPGA board to validate the RWU-RV64I core operation in hardware.
  \item Evaluation of program size and performance across different compiler optimization levels (\texttt{-O0}, \texttt{-O1}, \texttt{-O2}, \texttt{-O3}, \texttt{-Os}, and \texttt{-Ofast}).
\end{itemize}

It is important to note that this thesis does not include the design or modification of the processor’s RTL description, nor the development of a dedicated program loader.  
The scope is restricted to software-level development and integration, assuming that the RWU-RV64I processor and its memory interfaces are available and functionally verified.  
Furthermore, the toolchain targets a \emph{bare-metal} environment and therefore excludes any operating system or standard C library such as Newlib.  
All compiled programs are directly linked to the hardware memory map and executed without runtime dependencies.


\section{Objectives}
The main objective of this thesis is to establish a complete and functional software development environment for the RWU-RV64I processor. The defined scope forms the foundation for the following objectives, which translate the identified problems into concrete implementation goals.

The specific technical goals are summarized as follows:
\begin{itemize}
  \item Implement and configure the cross-compiler and associated binary utilities for the RV64I architecture.
  \item Create startup code (\texttt{crt0.s}) and a linker script tailored to the RWU-RV64I memory map and peripheral configuration.
  \item Integrate the toolchain with a structured build system and provide Eclipse IDE support for program compilation and debugging.
  \item Deploy and validate the compiled programs on the Zybo Zynq-7000 FPGA board to verify the functional operation of the RWU-RV64I core in hardware.
  \item Evaluate and compare program size across various compiler optimization levels (\texttt{-O0}, \texttt{-O1}, \texttt{-O2}, \texttt{-O3}, \texttt{-Os}, and \texttt{-Ofast}).
\end{itemize}

The successful completion of these objectives will result in a reproducible, maintainable, and extensible development workflow that allows C programs to be built, tested, and executed efficiently on the RWU-RV64I processor—both in simulation and on FPGA-based hardware platforms.

\begin{abstract}
\addcontentsline{toc}{chapter}{Abstract}

The rapid evolution of open-source processor architectures, particularly RISC-V, has enabled academic institutions and research groups to explore customized hardware software co-designs. This thesis presents the development of a \textbf{complete software toolchain} for the \textbf{RWU-RV64I}, a 64-bit RISC-V based in-house processor developed at \university. The work focuses on establishing a professional, extensible, and automated development environment that bridges the gap between hardware implementation and software programmability for the RWU-RV64I core.

The project begins with a detailed study of the existing prototype environment, which comprised a basic Makefile-based build system with limited support for assembly-level testing. To advance beyond this stage, the toolchain was redesigned and implemented around the \textbf{GNU Compiler Collection (GCC)} and a \textbf{custom linker script and startup code} tailored for the RWU-RV64I memory map and hardware configuration. The integration ensures that standard C programs can be compiled, linked, and executed directly on the custom processor without modification of its RTL design.

The thesis further introduces a \textbf{structured build framework} using \textbf{Make}, providing a modular and portable foundation for future development. This framework automates compilation, linking, and binary generation, ensuring reproducibility and ease of use for both academic research and student projects. Additionally, the implementation of test programs and GPIO verification routines demonstrates the functional correctness of the toolchain in simulation environments. The toolchain was also extended to support \textbf{FPGA-based deployment} on the \textbf{Zybo Zynq-7000} board, enabling real hardware validation.

Comprehensive testing confirmed successful compilation and execution of bare-metal C programs, verifying correct instruction decoding, memory mapping, and I/O control through GPIO signals. The developed infrastructure not only simplifies software development for the RWU-RV64I but also serves as a scalable foundation for future enhancements such as interrupt handling, UART communication, and operating system porting.

Overall, this thesis contributes a fully functional, reproducible, and extensible toolchain for the RWU-RV64I processor, providing a complete workflow from source code to hardware execution. It strengthens the processor’s usability as a research and educational platform and establishes a solid basis for ongoing developments in RISC-V–based embedded systems at \university.

\end{abstract}
